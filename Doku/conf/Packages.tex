%--------------------------------------------------------- INFORMATIONEN ----------------------------------------------------------------------------------------------------------

% 	Definition von globalen Parametern, die im gesamten Dokument verwendet
% 	werden können (z.B auf dem Deckblatt etc.).
%-----------------------------------------------------------------------------------------------------------------------------------------------------------------------------------------



\usepackage{cite}								% für Literaturverzeichnis
\usepackage{tocbasic}

%--------------------------------------------- FORMATIERUNG KOPF- UND FUSSZEILE -------------------------------------------------------------------------------------------

\usepackage									%
[automark,									% Automatische Kopfzeile
%headtopline,								% Linie über dem Seitenkopf
%plainheadtopline,								% Plain, Linie über dem Seitenkopf
headsepline,									% Linie zwischen Kopf und Textkörper
ilines,										% Trennlinie linksbündig ausrichten
%plainheadsepline,								% Plain, Linie zwischen Kopf und Textkörper
footsepline,									% Linie zwischen Textkörper und Fuss
plainfootsepline,   								% Plain, Linie zwischen Textkörper und Fuss
%footbotline,								% Linie unter dem Fuss
%plainfootbotline   								% Plain, Linie unter dem Fuss
]{scrpage2}									%

%------------------------------------------------------------------------------------------------------------------------------------------------------------------------------------------


%------------------------------------------------------ ANPASSUNG LANDESSPRACHE----------------------------------------------------------------------------------------------

% Verwendet globale Option german siehe \documentclass
% \usepackage{babel}
\usepackage{ngerman}

%------------------------------------------------------------------------------------------------------------------------------------------------------------------------------------------


%--------------------------------------------------------------------- UMLAUTE--------------------------------------------------------------------------------------------------------

\usepackage[T1]{fontenc}						% Dadurch ist auch sichergestellt, dass in einem PDF Umlaute gefunden werden.
\usepackage[utf8]{inputenc}						% Unterstützung erweiterter Zeichensätze mit  unterschiedlichen Kodierungen
										% (z. B. ä, ö, ü usw.)		|

%\usepackage{ae} 							% "schöneres" ö
\usepackage{textcomp} 							% Euro-Zeichen etc.

% @inputenc:
% inputenc (Einbindung mit \usepackage[latin1]{inputenc}) ist für die Unterstützung erweiterter Eingabe-Zeichensätze mit ihren unterschiedlichen
% Kodierungen (z. B. ä, ö, ü usw.). Es dient der Umwandlung einer beliebigen Eingabe-Zeichenkodierung in eine interne LaTeX-Standardsprache.
% Neben latin1 gibt es noch weitere Codierungen, so zum Beispiel applemac (eine Kodierung, die häufig für Textdateien unter Mac OS verwendet wurde)
% oder utf8 für UTF-8-kodierte Dateien. Im Austausch kann es aufgrund der unterschiedlichen Eingabe-Zeichenkodierung zu Problemen kommen.

%------------------------------------------------------------------------------------------------------------------------------------------------------------------------------------------

\usepackage{pifont}							%Sonderzeichen

%---------------------------------------------------------------EIGENE REFERENZIERUNG-------------------------------------------------------------------------------------------

\newcommand{\refTC}[1]{ \ref{#1} \nameref{#1}  }		% gibt aus z.B.: 1.1.3 Erwartetes Resultat (<Kapitelnummer> <Kapitelname>)

%------------------------------------------------------------------------------------------------------------------------------------------------------------------------------------------



%---------------------------------------------------------------------- GRAFIKEN------------------------------------------------------------------------------------------------------

\usepackage[dvips,final]{graphicx}					% Einbinden von EPS-Grafiken [draft oder final]
\graphicspath{{pictures/}} 						% Hier liegen die Bilder des Dokuments

%------------------------------------------------------------------------------------------------------------------------------------------------------------------------------------------


\usepackage{amsmath,amsfonts}					% Befehle aus AMSTeX für mathematische Symbole z.B. \boldsymbol \mathbb

\usepackage{multirow}							% Packages für Tabellen

% Weitere Zeichen z.B. \textcelsius \textordmasculine \textsurd \textonehalf 
% \texteuro \texttimes \textdiv ... aus textcomp.sty
% siehe >>Schnell ans Ziel mit \LaTeXe<< von Jörg Knappen
% (Oldenbourg, München und Wien 1997, ISBN 3-486-24199-0)
% \usepackage{tccompat}
	
%---------------------------------------------------------------- OWN COLORS-------------------------------------------------------------------------------------------------------

%************************************************************************************
%*                                                                      Farbdefinitionen                                                                    *
%*                                                                                nach                                                                            *
%*                                                        http://www.tayloredmktg.com/rgb/                                                   *
%************************************************************************************

% \definecolor{}{rgb}{}
\usepackage{color}

%************************************ Whites/Pastels ***********************************

\definecolor{Snow}{rgb}{1.000,0.980,0.980}
\definecolor{Snow2}{rgb}{0.933,0.913,0.913}
\definecolor{Snow3}{rgb}{0.804,0.788,0.788}
\definecolor{Snow4}{rgb}{0.545,0.537,0.537}
\definecolor{GhostWhite}{rgb}{0.973,0.973,1}
\definecolor{WhiteSmoke}{rgb}{0.961,0.961,0.961}
\definecolor{Gainsboro}{rgb}{0.863,0.863,0.863}
\definecolor{FloralWhite}{rgb}{1,0.980,0.941}
\definecolor{OldLace}{rgb}{0.992,0.961,0.902}
\definecolor{Linen}{rgb}{0.941,0.941,0.902}
\definecolor{AntiqueWhite}{rgb}{0.980,0.922,0.843}
\definecolor{AntiqueWhite2}{rgb}{0.933,0.875,0.800}
\definecolor{AntiqueWhite3}{rgb}{0.804,0.753,0.690}
\definecolor{AntiqueWhite4}{rgb}{0.545,0.514,0.471}
\definecolor{PapayaWhip}{rgb}{1,0.937,0.835}
\definecolor{BlanchedAlmond}{rgb}{1,0.922,0.804}
\definecolor{Bisque}{rgb}{1,0.894,0.769}
\definecolor{Bisque2}{rgb}{0.933,0.835,0.718}
\definecolor{Bisque3}{rgb}{0.804,0.718,0.620}
\definecolor{Bisque4}{rgb}{0.545,0.490,0.420}
\definecolor{PeachPuff}{rgb}{1,0.855,0.725}
\definecolor{PeachPuff2}{rgb}{0.933,0.796,0.678}
\definecolor{PeachPuff3}{rgb}{0.804,0.686,0.584}
\definecolor{PeachPuff4}{rgb}{0.545,0.467,0.396}
\definecolor{NavajoWhite}{rgb}{1,0.871,0.678}
\definecolor{Moccasin}{rgb}{1,0.894,0.710}
\definecolor{Cornsilk}{rgb}{1,0.973,0.863}
\definecolor{Cornsilk2}{rgb}{0.933,0.910,0.804}
\definecolor{Cornsilk3}{rgb}{0.804,0.784,0.694}
\definecolor{Cornsilk4}{rgb}{0.545,0.533,0.471}
\definecolor{Ivory}{rgb}{1,1,0.941}
\definecolor{Ivory2}{rgb}{0.933,0.933,0.878}
\definecolor{Ivory3}{rgb}{0.804,0.804,0.757}
\definecolor{Ivory4}{rgb}{0.545,0.545,0.514}
\definecolor{LemonChiffon}{rgb}{1,0.980,0.804}
\definecolor{Seashell}{rgb}{1,0.961,0.933}
\definecolor{Seashell2}{rgb}{0.933,0.898,0.871}
\definecolor{Seashell3}{rgb}{0.804,0.773,0.749}
\definecolor{Seashell4}{rgb}{0.545,0.525,0.510}
\definecolor{Honeydew}{rgb}{0.941,1,0.941}
\definecolor{Honeydew2}{rgb}{0.957,0.933,0.878}
\definecolor{Honeydew3}{rgb}{0.757,0.804,0.757}
\definecolor{Honeydew4}{rgb}{0.514,0.545,0.514}
\definecolor{MintCream}{rgb}{0.957,1,0.980}
\definecolor{Azure}{rgb}{0.941,1,1}
\definecolor{AliceBlue}{rgb}{0.941,0.973,1}
\definecolor{Lavender}{rgb}{0.902,0.902,0.980}
\definecolor{LavenderBlush}{rgb}{1,0.941,0.961}
\definecolor{MistyRose}{rgb}{1,0.894,0.882}
\definecolor{White}{rgb}{1,1,1}

%************************************ Grays ***********************************

\definecolor{Black}{rgb}{0,0,0}
\definecolor{DarkSlateGray}{rgb}{0.192,0.310,0.310}
\definecolor{DimGray}{rgb}{0.412,0.412,0.412}
\definecolor{SlateGray}{rgb}{0.439,0.541,0.565}
\definecolor{LightSlateGray}{rgb}{0.467,0.533,0.600}
\definecolor{Gray}{rgb}{0.745,0.745,0.745}
\definecolor{LightGray}{rgb}{0.827,0.827,0.827}

%************************************ Blues ***********************************

\definecolor{MidnightBlue}{rgb}{0.098,0.098,0.439}
\definecolor{Navy}{rgb}{0,0,0.502}
\definecolor{CornflowerBlue}{rgb}{0.392,0.584,0.929}
\definecolor{DarkSlateBlue}{rgb}{0.282,0.239,0.545}
\definecolor{SlateBlue}{rgb}{0.416,0.353,0.804}
\definecolor{MediumSlateBlue}{rgb}{0.482,0.408,0.933}
\definecolor{LightSlateBlue}{rgb}{0.518,0.439,1}
\definecolor{MediumBlue}{rgb}{0,0,0.804}
\definecolor{RoyalBlue}{rgb}{0.255,0.412,0.882}
\definecolor{Blue}{rgb}{0,0,1}
\definecolor{DodgerBlue}{rgb}{0.118,0.565,1}
\definecolor{DeepSkyBlue}{rgb}{0,0.749,1}
\definecolor{SkyBlue}{rgb}{0.529,0.808,0.980}
\definecolor{SteelBlue}{rgb}{0.275,0.510,0.706}
\definecolor{LightSteelBlue}{rgb}{0.690,0.769,0.871}
\definecolor{LightBlue}{rgb}{0.678,0.847,0.902}
\definecolor{PowderBlue}{rgb}{0.690,0.878,0.902}
\definecolor{PaleTurquoise}{rgb}{0.686,0.933,0.933}
\definecolor{DarkTurquoise}{rgb}{0,0.808,0.820}
\definecolor{MediumTurquoise}{rgb}{0.282,0.820,0.800}
\definecolor{Turquoise}{rgb}{0.251,0.878,0.816}
\definecolor{Cyan}{rgb}{0,1,1}
\definecolor{LightCyan}{rgb}{0.878,1,1}
\definecolor{CadetBlue}{rgb}{0.373,0.620,0.627}

%************************************ Greens ***********************************

\definecolor{MediumAquamarine}{rgb}{0.400,0.804,0.667}
\definecolor{Aquamarine}{rgb}{0.498,1,0.831}
\definecolor{DarkGreen}{rgb}{0,0.392,0}
\definecolor{DarkOliveGreen}{rgb}{0.333,0.420,0.184}
\definecolor{DarkSeaGreen}{rgb}{0.561,0.737,0.561}
\definecolor{SeaGreen}{rgb}{0.180,0.545,0.341}
\definecolor{MediumSeaGreen}{rgb}{0.235,0.702,0.443}
\definecolor{LightSeaGreen}{rgb}{0.125,0.698,0.667}
\definecolor{PaleGreen}{rgb}{0.596,0.984,0.596}
\definecolor{SpringGreen}{rgb}{0,1,0.498}
\definecolor{LawnGreen}{rgb}{0.486,0.988,0}
\definecolor{Chartreuse}{rgb}{0.498,1,0}
\definecolor{MediumSpringGreen}{rgb}{0,0.980,0.604}
\definecolor{GreenYellow}{rgb}{0.678,1,0.184}
\definecolor{LimeGreen}{rgb}{0.196,0.804,0.196}
\definecolor{YellowGreen}{rgb}{0.604,0.804,0.196}
\definecolor{ForestGreen}{rgb}{0.133,0.545,0.133}
\definecolor{OliveDrab}{rgb}{0.420,0.557,0.137}
\definecolor{DarkKhaki}{rgb}{0.741,0.718,0.420}
\definecolor{Khaki}{rgb}{0.941,0.902,0.549}

%************************************ Yellow ***********************************

\definecolor{PaleGoldenrod}{rgb}{0.933,0.910,0.667}
\definecolor{LightGoldenrodYellow}{rgb}{0.980,0.980,0.824}
\definecolor{LightYellow}{rgb}{1,1,0.878}
\definecolor{Yellow}{rgb}{1,1,0}
\definecolor{Gold}{rgb}{1,0.843,0}
\definecolor{LightGoldenrod}{rgb}{0.933,0.867,0.510}
\definecolor{Goldenrod}{rgb}{0.855,0.647,0.125}
\definecolor{DarkGoldenrod}{rgb}{0.722,0.525,0.043}

%************************************ Browns ***********************************

\definecolor{RosyBrown}{rgb}{0.737,0.561,0.561}
\definecolor{IndianRed}{rgb}{0.804,0.361,0.361}
\definecolor{SaddleBrown}{rgb}{0.545,0.271,0.075}
\definecolor{Sienna}{rgb}{0.627,0.322,0.176}
\definecolor{Peru}{rgb}{0.804,0.522,0.247}
\definecolor{Burlywood}{rgb}{0.871,0.722,0.529}
\definecolor{Beige}{rgb}{0.961,0.961,0.863}
\definecolor{Wheat}{rgb}{0.961,0.871,0.702}
\definecolor{SandyBrown}{rgb}{0.957,0.643,0.376}
\definecolor{Tan}{rgb}{0.824,0.706,0.549}
\definecolor{Chocolate}{rgb}{0.824,0.412,0.118}
\definecolor{Firebrick}{rgb}{0.698,0.133,0.133}
\definecolor{Brown}{rgb}{0.647,0.165,0.165}

%************************************ Oranges ***********************************

\definecolor{DarkSalmon}{rgb}{0.914,0.588,0.478}
\definecolor{Salmon}{rgb}{0.980,0.502,0.447}
\definecolor{LightSalmon}{rgb}{1,0.627,0.478}
\definecolor{Orange}{rgb}{1,0.647,0}
\definecolor{DarkOrange}{rgb}{1,0.549,0}
\definecolor{Coral}{rgb}{1,0.498,0.314}
\definecolor{LightCoral}{rgb}{0.941,0.502,0.502}
\definecolor{Tomato}{rgb}{1,0.388,0.278}
\definecolor{OrangeRed}{rgb}{1,0.271,0}
\definecolor{Red}{rgb}{1,0,0}

%************************************ Pinks/Violets ***********************************

\definecolor{HotPink}{rgb}{1,0.412,0.706}
\definecolor{DeepPink}{rgb}{1,0.078,0.576}
\definecolor{Pink}{rgb}{1,0.753,0.796}
\definecolor{LightPink}{rgb}{1,0.714,0.757}
\definecolor{PaleVioletRed}{rgb}{0.859,0.439,0.576}
\definecolor{Maroon}{rgb}{0.690,0.188,0.376}
\definecolor{MediumVioletRed}{rgb}{0.780,0.082,0.522}
\definecolor{VioletRed}{rgb}{0.816,0.125,0.565}
\definecolor{Violet}{rgb}{0.933,0.510,0.933}
\definecolor{Plum}{rgb}{0.867,0.627,0.867}
\definecolor{Orchid}{rgb}{0.855,0.439,0.839}
\definecolor{MediumOrchid}{rgb}{0.729,0.333,0.827}
\definecolor{DarkOrchid}{rgb}{0.6,0.196,0.8}
\definecolor{DarkViolet}{rgb}{0.580,0,0.827}
\definecolor{BlueViolet}{rgb}{0.541,0.169,0.886}
\definecolor{Purple}{rgb}{0.627,0.125,0.941}
\definecolor{MediumPurple}{rgb}{0.576,0.439,0.859}
\definecolor{Thistle}{rgb}{0.847,0.749,0.847}							% Eigene Farbdefinitionen nach RGB Chart aus Internet 

%------------------------------------------------------------------------------------------------------------------------------------------------------------------------------------------


\usepackage{makeidx}							% Für Index-Ausgabe; \printindex

%-------------------------------------------------------------------------- QUELLCODE AUSGABE------------------------------------------------------------------------------------

%************************************************************************************
%*                                                    Definitionen zur Quellcode Darstellung                                                   *
%************************************************************************************

\usepackage{listings}

\lstset {language=C++,
	keywordstyle={\color{Blue}\bfseries},
	commentstyle={\color{ForestGreen}\slshape},
	stringstyle={\color{Purple}},
	backgroundcolor={\color{PowderBlue}},
	showstringspaces=fales,
	stepnumber=2,
	numbers=left,
	numberstyle=\tiny}

\lstset {language=SQL,
	keywordstyle={\color{IndianRed}\bfseries},
	commentstyle={\color{ForestGreen}\slshape},
	identifierstyle=\ttfamily\color{CadetBlue}\bfseries,
	stringstyle={\color{Purple}},
	backgroundcolor={\color{Tan}},
	showstringspaces=fales,
	stepnumber=2,
	numbers=left,
	numberstyle=\tiny}								% Eigene Ausgabeformatierungen für Quellcode

%------------------------------------------------------------------------------------------------------------------------------------------------------------------------------------------


%------------------------------------------------------------------ ZEILENABSTÄNDE UND SEITENRÄNDER----------------------------------------------------------------------


% Einfache Definition der Zeilenabstände und Seitenränder etc. -------------
\usepackage{setspace}							% hat zur Benützung 3 Optionen: [singlespacing, onehalfspacing, doublespacing]
\usepackage{geometry}	


% Symbolverzeichnis --------------------------------------------------------
% 	Symbolverzeichnisse bequem erstellen, beruht auf MakeIndex.
% 		makeindex.exe %Name%.nlo -s nomencl.ist -o %Name%.nls
% 	erzeugt dann das Verzeichnis. Dieser Befehl kann z.B. im TeXnicCenter
%		als Postprozessor eingetragen werden, damit er nicht ständig manuell
%		ausgeführt werden muss.
%		Die Definitionen sind ausgegliedert in die Datei Abkuerzungen.tex.
% --------------------------------------------------------------------------
\usepackage[intoc]{nomencl}
  \let\abbrev\nomenclature
  \renewcommand{\nomname}{Abkürzungsverzeichnis}
  \setlength{\nomlabelwidth}{.25\hsize}
  \renewcommand{\nomlabel}[1]{#1 \dotfill}
  \setlength{\nomitemsep}{-\parsep}
  

%inserted november 2012 / Micha
\usepackage[normalem]{ulem}
\newcommand{\markup}[1]{\uline{#1}}



\usepackage{floatflt}							% Zum Umfließen von Bildern


% Zum Einbinden von Programmcode --------------------------------------------
\usepackage{listings}
\usepackage{xcolor} 
\definecolor{hellgelb}{rgb}{1,1,0.9}
\definecolor{colKeys}{rgb}{0,0,1}
\definecolor{colIdentifier}{rgb}{0,0,0}
\definecolor{colComments}{rgb}{1,0,0}
\definecolor{colString}{rgb}{0,0.5,0}
\lstset{%
    float=hbp,%
    basicstyle=\texttt\small, %
    identifierstyle=\color{colIdentifier}, %
    keywordstyle=\color{colKeys}, %
    stringstyle=\color{colString}, %
    commentstyle=\color{colComments}, %
    columns=flexible, %
    tabsize=2, %
    frame=single, %
    extendedchars=true, %
    showspaces=false, %
    showstringspaces=false, %
    numbers=left, %
    numberstyle=\tiny, %
    breaklines=true, %
    backgroundcolor=\color{hellgelb}, %
    breakautoindent=true, %
%    captionpos=b%
}

% Lange URLs umbrechen etc. -------------------------------------------------
\usepackage[hyphens]{url}						% bricht lange URL's um, ohne dass der Link defekt ist
\usepackage{url}




% Wichtig für korrekte Zitierweise ------------------------------------------
%%\usepackage[square]{natbib}
% Quellenangaben in eckige Klammern fassen ---------------------------------
%%\bibpunct{[}{]}{;}{a}{}{,~}


%\usepackage{jurabib}
%\jurabibsetup{authorformat=smallcaps,% Autor in Kapitälchen              
%              %authorformat=year,
%              authorformat=citationreversed,% Im Zitat Vorname vorne
%              authorformat=indexed,% Autor in Index
%              authorformat=and,% Autoren mit "," und "und" abgetrennt
%              authorformat=firstnotreversed,%
%              authorformat=reducedifibidem,% Bei Verweis nur Nachname. 
%              %superscriptedition=all,% Auflage hochgestellt
%              %citefull=first,% Erstzitat voll
%              titleformat=italic,              
%              titleformat=all,
%              titleformat=colonsep,% Doppelpunkt zwischen Aut. u. Titel
%              ibidem=strict,% Ebenda pro Doppelseite
%              see,% Das zweite Argument ist optional für "Vgl." etc.
%              commabeforerest,% Komma vor Seitenzahl
%              %howcited=compare,%
%              %bibformat=ibidem,% Strich bei widerholtem Autor in BIB.
%              commabeforerest,
%              bibformat=compress,
%              pages=always,
%              %pages=format,% S. wird vorweggesetzt
%              crossref=long,% Querverweise in voller Länge
%              square,% eckige Klammern bei Zitaten
%              %oxford,
%              %chicago,
%}
%
%\AddTo\bibsgerman{% 
%\jblookforgender%
%\renewcommand*{\ibidemname}{Ebenda}%
%\renewcommand*{\ibidemmidname}{ebenda}% 
%\renewcommand*{\bibjtsep}{In: }% Vor Zeitschriften 
%\renewcommand*{\bibbtsep}{In: }% Vor Buchtitel
%\renewcommand*{\incollinname}{In: }%Nicht so ganz sauber. 
%\renewcommand*{\bibatsep}{.}% Nach Titel
%\renewcommand*{\bibbdsep}{}%Vor Datum 
%\renewcommand*{\jbaensep}{.}%
%\renewcommand*{\bibprdelim}{)}% Klammer bei Zeitschriftjahr rechts
%\renewcommand*{\bibpldelim}{(}% Klammer bei Zeitschriftjahr links
%\renewcommand*{\biblnfont}{\textsc}% Nachamen Autor im BIB
%\renewcommand*{\bibelnfont}{\textsc}% Nachamen Hg. im BIB
%\renewcommand*{\bibfnfont}{\textsc}% Vorn. Autor im BIB
%\renewcommand*{\bibefnfont}{\textsc}% Vorn. Hg. im BIB
%\renewcommand*{\jbcitationyearformat}[1]{#1}% Komma zwischen Autor und Jahr entfernen
%\def\herename{hier: }%
%\jbfirstcitepageranges% Format: S. x--z, hier y.  
%\renewcommand\bibidemSfname{\raisebox{.2em}{\rule{2.em}{.2pt}}~}%
%\renewcommand\bibidemsfname{\raisebox{.2em}{\rule{2.em}{.2pt}}~}%
%\renewcommand\bibidemPfname{\raisebox{.2em}{\rule{2.em}{.2pt}}~}%
%\renewcommand\bibidempfname{\raisebox{.2em}{\rule{2.em}{.2pt}}~}%
%\renewcommand\bibidemSmname{\raisebox{.2em}{\rule{2.em}{.2pt}}~}%
%\renewcommand\bibidemsmname{\raisebox{.2em}{\rule{2.em}{.2pt}}~}%
%\renewcommand\bibidemPmname{\raisebox{.2em}{\rule{2.em}{.2pt}}~}%
%\renewcommand\bibidempmname{\raisebox{.2em}{\rule{2.em}{.2pt}}~}%
%\renewcommand\idemSfname{Dies.}%
%\renewcommand\idemsfname{dies.}%
%\renewcommand\idemPfname{Dies.}%
%\renewcommand\idempfname{dies.}%
%\renewcommand\idemSmname{Ders.}%
%\renewcommand\idemsmname{ders.}%
%\renewcommand\idemPmname{Dies.}%
%\renewcommand\idempmname{dies.}%
%\renewcommand{\jbannoteformat}[1]{{\footnotesize\begin{quote}#1\end{quote}}}
%}%


% --------------------------------------------------------------------------PDF OPTIONEN--------------------------------------------------------------------------------------------
								
\usepackage[								%
bookmarks,									%
bookmarksopen=true,							%
pdftitle={\titel},								%
pdfauthor={\autorFirst},							%
pdfcreator={\autorFirst},							%
pdfsubject={\titel},								%
pdfkeywords={\titel},							%
colorlinks=true,								%
linkcolor=red, 								% einfache interne Verknüpfungen
anchorcolor=black,								% Ankertext
citecolor=blue, 								% Verweise auf Literaturverzeichniseinträge im Text
filecolor=magenta, 								% Verknüpfungen, die lokale Dateien öffnen
menucolor=red, 								% Acrobat-Menüpunkte
urlcolor=cyan, 								%
%linkcolor=black, 								% einffdrtext
%citecolor=black, 								% Verweise auf Literaturverzeichniseinträge im Text
%filecolor=black, 								% Verknüpfungen, die lokale Dateien öffnen
%menucolor=black,							% Acrobat-Menüpunkte
%urlcolor=black, 								%
backref,									%
%pagebackref,								%
plainpages=false,								% zur korrekten Erstellung der Bookmarks
pdfpagelabels,								% zur korrekten Erstellung der Bookmarks
hypertexnames=false,							% zur korrekten Erstellung der Bookmarks
linktocpage 									% Seitenzahlen anstatt Text im Inhaltsverzeichnis verlinken
]{hyperref}									%

\usepackage[final]{pdfpages}			%Ermöglicht andere PDF ins Dokument einzufügen.

%------------------------------------------------------------------------------------------------------------------------------------------------------------------------------------------


% Zum fortlaufenden Durchnummerieren der Fußnoten ---------------------------
\usepackage{chngcntr}


% Aliase für Zitate
% \defcitealias{WPProzess}{Wikipedia:~Prozess}

%\usepackage{minitoc}

% für lange Tabellen
\usepackage{longtable}
\usepackage{array}
\usepackage{ragged2e}
\usepackage{lscape}

%Spaltendefinition rechtsbündig mit definierter Breite
\newcolumntype{w}[1]{>{\raggedleft\hspace{0pt}}p{#1}}

% Formatierung von Listen ändern
\usepackage{paralist}
% \setdefaultleftmargin{2.5em}{2.2em}{1.87em}{1.7em}{1em}{1em}



% Anhangsverzeichnis
%\makeatletter% --> De-TeX-FAQ
%\newcommand*{\maintoc}{% Hauptinhaltsverzeichnis
%\begingroup
%\@fileswfalse% kein neues Verzeichnis öffnen
%\renewcommand*{\appendixattoc}{% Trennanweisung im Inhaltsverzeichnis
%\value{tocdepth}=-10000 % lokal tocdepth auf sehr kleinen Wert setzen
%}%
%\tableofcontents% Verzeichnis ausgeben
%\endgroup
%}
%\newcommand*{\appendixtoc}{% Anhangsinhaltsverzeichnis
%\begingroup
%\edef\@alltocdepth{\the\value{tocdepth}}% tocdepth merken
%\setcounter{tocdepth}{-10000}% Keine Verzeichniseinträge
%\renewcommand*{\contentsname}{% Verzeichnisname ändern
%Verzeichnis der Anh\"ange}%
%\renewcommand*{\appendixattoc}{% Trennanweisung im Inhaltsverzeichnis
%\setcounter{tocdepth}{\@alltocdepth}% tocdepth wiederherstellen
%}%
%\tableofcontents% Verzeichnis ausgeben
%\setcounter{tocdepth}{\@alltocdepth}% tocdepth wiederherstellen
%\endgroup
%}
%\newcommand*{\appendixattoc}
%\g@addto@macro\appendix{% \appendix erweitern
%\if@openright\cleardoublepage\else\clearpage\fi% Neue Seite
%\addcontentsline{toc}{chapter}{\appendixname}% Eintrag ins Hauptverzeichnis
%\addtocontents{toc}{\protect\appendixattoc}% Trennanweisung in die toc-Datei
%}
%\makeatother
